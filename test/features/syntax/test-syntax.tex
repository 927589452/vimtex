\documentclass{beamer}
\usepackage{hyperref}
\usepackage{array}
\usepackage{amsmath}
\usepackage{tabularx}
\usepackage{cleveref}
\usepackage{varioref}
\usepackage{listings}
\usepackage{minted}
\usepackage{dot2texi}
\usepackage{gnuplottex}
\usepackage[inline]{asymptote}

%
% Test (re)newcommand and (re)newenvironment
%
\newcommand{\testA}[1]{Test #1}
\renewcommand{\testA}[1]{Test #1}
\newenvironment{example}[1][]{Start #1}{Stop}
\renewenvironment{example}[1][]{Start #1}{Stop}

%
% Test nested (re)newcommand and (re)newenvironment
%
\newcommand{\mycommand}[1]{%
  \renewcommand{\myothercommand}[1]{%
    \dosomethingwithouterarg{#1}%
    \dosomethingwithinnerarg{##1}%
  }%
}
\newenvironment{nestedexample}[1][]{%
  \renewenvironment{nested}[1][0]{%
    Outerarg #1}{%
    Innerarg ##1
  }%
  \begin{nested}{title}}{%
  \end{nested}%
}

\begin{document}

\section{Miscellaneous}
\def\A{A}
\def\B{\V{B}}

\section{Documentclass: beamer}
\begin{frame}[asd]
  \includegraphics[height=0.5062\linewidth]{test.png}
  \includegraphics<2>[height=0.5062\linewidth]{test.png}

  \only{asd}
  \only<2>{asd}

  \item asdasd
  \item<1-3> asdasd
\end{frame}

\section{Package: hyperref and url}
\url{http://www.google.com}
\url+http://www.google.com+
\href{http://example.com}{Title text $with math$}
\urldef{\mysite}\url{http://example.com}
\hyperref[asdasd]{asd}

\section{Package: biblatex and csquotes}
\cite{}
\citet*{}
\citep{bibtexkey1}
\citep[e.g.][]{bibtexkey2}
\citealt{}
\citealt*{}
\citealp{}
\citealp*{}
\citenum{}
\citetext{}
\citeauthor{}
\citeauthor*{}
\citeyear{}
\citeyearpar{}
\bibentry{}

\section{Package: cleveref}
\label[asd]{sec:cleveref}

\crefname{equation}{Eq}{Eqs}
\cref{eq:test}
\crefrange{eq:test}{eq:test2}

\section{Package: tabularx and array}
\begin{tabular}{*{3}{>{$}c<{$}}}
  \hline
  Hello world !\tabularnewline
  \hline
\end{tabular}

\begin{tabular}{l*{4}{m{1cm}}}
  \hline
  Hello world !\tabularnewline
  \hline
\end{tabular}

\begin{tabular}[ht]{|>{\centering$}@{.}m{1cm}<{$}|}
  \hline
  Hello world !\tabularnewline
  \hline
\end{tabular}

\section{Package: varioref}
\Vref{sec_one} does what you need. Still \vref{sec_one} works.

\section{Package: amsmath}
\begin{equation}
  f(x) = 0
\end{equation}

\begin{align}
  f(x) = 0
\end{align}

\section{Package: dot2texi}
\begin{dot2tex}
  graph graphname {
    a -- b;
    b -- c;
    b -- d;
    d -- a;
  }
\end{dot2tex}

\section{Package: luacode}
\directlua{
  if pdf.getminorversion() \string~= 7 then
    print "pfd version 1.7"
  end
}

\begin{luacode}
  if pdf.getminorversion() \string~= 7 then
    print "pfd version 1.7"
  end
\end{luacode}

\section{Package: gnuplottex}
\begin{gnuplot}[terminal=..., terminaloptions=...]
    plot file using 1:2 with lines
\end{gnuplot}

\section{Package: asymptote}
Here is a venn diagram produced with Asymptote, drawn to width 4cm:

\begin{asy}
  size(4cm,0);
  pen colour1=red;
  pen colour2=green;

  pair z0=(0,0);
  pair z1=(-1,0);
  pair z2=(1,0);
  real r=1.5;
  path c1=circle(z1,r);
  path c2=circle(z2,r);
  fill(c1,colour1);
  fill(c2,colour2);

  picture intersection=new picture;
  fill(intersection,c1,colour1+colour2);
  clip(intersection,c2);

  add(intersection);

  draw(c1);
  draw(c2);

  // draw("$\A$",box,z1);              // Requires [inline] package option.
  // draw(Label("$\B$","$B$"),box,z2); // Requires [inline] package option.
  draw("$A$",box,z1);
  draw("$\V{B}$",box,z2);

  pair z=(0,-2);
  real m=3;
  margin BigMargin=Margin(0,m*dot(unit(z1-z),unit(z0-z)));

  draw(Label("$A\cap B$",0),conj(z)--z0,Arrow,BigMargin);
  draw(Label("$A\cup B$",0),z--z0,Arrow,BigMargin);
  draw(z--z1,Arrow,Margin(0,m));
  draw(z--z2,Arrow,Margin(0,m));

  shipout(bbox(0.25cm));
\end{asy}

\begin{asydef}
  // Global Asymptote definitions can be put here.
  import three;
  usepackage("bm");
  texpreamble("\def\V#1{\bm{#1}}");
  // One can globally override the default toolbar settings here:
  // settings.toolbar=true;
\end{asydef}

\section{Package: listings}
\begin{lstlisting}
testing
\end{lstlisting}

\begin{lstlisting}[directivestyle={\color{black}},
                   language=C,
                   emph={int,char,double,float,unsigned},
                   emphstyle={\color{blue}}]
#include<stdio.h>
main() {
  int i = 0;
  printf("Hello World");
}
\end{lstlisting}

\begin{lstlisting}[language=python]
def function(arg):
    pass
\end{lstlisting}

\lstset{language=rust,numbers=none}
\begin{lstlisting}
fn main() {
    println!("Hello World!");
}
\end{lstlisting}

\section{Package: minted}
\begin{minted}{python}
def function(arg):
    pass
\end{minted}

\begin{cppcode}
int main() {
  printf("hello, world");
  return 0;
}
\end{cppcode}

\begin{cppcode_test}
int main() {
  printf("hello, world");
  return 0;
}
\end{cppcode_test}

\begin{cppcode*}{a=0,
                 b=1}
int main() {
  printf("hello, world");
  return 0;
}
\end{cppcode*}

\begin{minted}
  L1045 +++$+++ u0 +++$+++ m0 +++$+++ BIANCA +++$+++ They do not!
  L1044 +++$+++ u0 +++$+++ m0 +++$+++ CAMERON +++$+++ They do to!
  L985 +++$+++ u0 +++$+++ m0 +++$+++ BIANCA +++$+++ I hope so.!
\end{minted}

\begin{minted}{c}
int main() {
  printf("hello, world");
  return 0;
}
\end{minted}

\begin{minted}[mathescape,
               linenos,
               numbersep=5pt,
               gobble=2,
               frame=lines,
               framesep=2mm]{csharp}
string title = "This is a Unicode π in the sky"
/*
Defined as $\pi=\lim_{n\to\infty}\frac{P_n}{d}$ where $P$ is the perimeter
of an $n$-sided regular polygon circumscribing a
circle of diameter $d$.
*/
const double pi = 3.1415926535
\end{minted}

\end{document}
