\documentclass{article}
\usepackage{amsmath}

\usepackage[backend=biber]{biblatex}
\addbibresource{biblatex-examples.bib}
\addbibresource{local1.bib}
\addbibresource{~/.vim/bundle/vim-latex/test/completion/local2.bib}

\usepackage[nonumberlist,acronymlists={gloss,symbolslist}]{glossaries}
\newglossary*{gloss}{Glossary}
\newglossary*{symbols}{List of symbols}

\makeglossaries

\newglossaryentry{kv}
{
  type=gloss,
  name=KV,
  description={Kontrollvolumen}
}

\newglossaryentry{dgl}
{
  type=gloss,
  name=DGL,
  description={Differentialgleichung}
}

\newglossaryentry{agl}
{
  type=gloss,
  name=AGL,
  description={Algebraische Gleichung}
}

\newglossaryentry{dichte}
{
  sort=1171,
  type=symbols,
  name={\ensuremath{\varrho}},
  description={
    Dichte, $[\varrho] = \frac{kg}{m^3}$
  }
}
\newglossaryentry{c}
{
  sort=0031,
  type=symbols,
  name={\ensuremath{c}},
  description={
    spezifische Wärmekapazität, $[c] = \frac{J}{kg\cdot K}$
  }
}
\newglossaryentry{C}
{
  sort=0030,
  type=symbols,
  name={\ensuremath{C_{th}}},
  description={
    Wärmekapazität, $[C_{th}] = \frac{J}{K}$
  }
}


\begin{document}

To test label completion you need to compile the document first!

\begin{equation}
  f(x) = 42
  \label{eq:main-is-working}
\end{equation}

\section{sub1}\label{sec:sub1}

\begin{equation}
  f(x) = 42
  \label{eq:sub-is-working}
\end{equation}


\include{"sub2\space with\space spaces"}

\input{"sub3\space with\space spaces"}

\begin{equation}
  \label{eq:test1}
  f(x) = 1
\end{equation}

\begin{equation}
  \label{eq:test2}
  f(x) = 2
\end{equation}

\begin{equation}
  \label{eq:test3}
  f(x) = 3
\end{equation}

\begin{equation}
  \label{eq:test4}
  f(x) = 4
\end{equation}

\begin{equation}
  \label{eq:test5}
  f(x) = 5
\end{equation}

\begin{equation}
  \label{eq:test6}
  f(x) = 6
\end{equation}

\begin{equation}
  \label{eq:test7}
  f(x) = 7
\end{equation}

\begin{equation}
  \label{eq:test8}
  f(x) = 8
\end{equation}

\begin{equation}
  \label{eq:test9}
  f(x) = 9
\end{equation}

\begin{equation}
  \label{eq:test10}
  f(x) = 10
\end{equation}

\begin{subequations}
  \begin{align}
    \label{eq:test11a}
    f(x) &= 11a \\
    \label{eq:test11b}
    f(x) &= 11b
  \end{align}
\end{subequations}

\newpage
\glsaddall
\printglossary[
  type=gloss,
  style=long,
  title={Glossary},
  toctitle={Glossary}
]
\printglossary[
  type=symbols,
  style=long,
  title={List of Symbols},
  toctitle={List of Symbols}
]

\end{document}
