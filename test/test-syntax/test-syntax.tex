\documentclass{article}
\usepackage{minted}
\begin{document}

\begin{lstlisting}
testing
\end{lstlisting}

\begin{align}
  f(x) = 0
\end{align}

\begin{equation}
  f(x) = 0
\end{equation}

\begin{quote}
  test
\end{quote}

\begin{dot2tex}
  graph graphname { 
    a -- b; 
    b -- c;
    b -- d;
    d -- a;
  }
\end{dot2tex}

\begin{minted}{python}
def function(arg):
    pass
\end{minted}

\begin{cppcode}
int main() {
  printf("hello, world");
  return 0;
}
\end{cppcode}

\begin{cppcode_test}
int main() {
  printf("hello, world");
  return 0;
}
\end{cppcode_test}

\begin{cppcode*}{a=0,
                 b=1}
int main() {
  printf("hello, world");
  return 0;
}
\end{cppcode*}

\begin{minted}{c}
int main() {
  printf("hello, world");
  return 0;
}
\end{minted}

\begin{minted}[mathescape,
               linenos,
               numbersep=5pt,
               gobble=2,
               frame=lines,
               framesep=2mm]{csharp}
string title = "This is a Unicode π in the sky"
/*
Defined as $\pi=\lim_{n\to\infty}\frac{P_n}{d}$ where $P$ is the perimeter
of an $n$-sided regular polygon circumscribing a
circle of diameter $d$.
*/
const double pi = 3.1415926535
\end{minted}

\end{document}
